\documentclass{article}
\usepackage{geometry}[a4paper, margin=1in]
\usepackage{amsmath}
\usepackage{amssymb}
\usepackage{graphicx}
\usepackage{diagbox}
\usepackage{subcaption}
\usepackage{float}
\usepackage{verbatim}

\title{Algorithms for Uncertainty Quantification\\
Bonus Assignment 2}
\author{Onur Yılmaz, 03764831 \\Zafer Bora Yılmazer, 03782277 }
\date{May 2024}

\begin{document}

\maketitle

\section{Lagrange Interpolation}
placeholder

\section{Orthogonal Polynomials}
\subsection{Orthogonal Check by Expectation Value}
By definition of the expectation of a continuous random variable (in our case $\rho$), the expected value is 
\begin{equation}
    \mathbb{E}[\phi_i(x)\cdot\phi_j(x)] = \int_{-\infty}^\infty\phi_i(x)\phi_j(x)\rho(x)dx
\end{equation}
From the definition of orthogonality of two polynomials, two polynomials are orthogonal if 
\begin{equation}
    \int_{-\infty}^\infty\phi_i(x)\phi_j(x)\rho(x)dx = \left\langle\phi_i(x), \phi_j(x)\right\rangle _\rho\delta_{ij}
\end{equation}
Therefore, after combining the two equations we get 
\begin{equation}
    \mathbb{E}[\phi_i(x)\cdot\phi_j(x)] = \left\langle\phi_i(x), \phi_j(x)\right\rangle _\rho\delta_{ij}
\end{equation}
which yields the following insights:
\begin{itemize}
    \item Expected value returns a non-zero value only if $i=j$ due to Kronecker delta.
    \item If the expected value is zero for all other $i$ and $j$ combinations, orthogonality between the two 
    polynomials is revealed.
    \item If the expected value is 1, one can say that the polynomials are orthonormal since 
    \[\int_{-\infty}^\infty\phi_i(x)\phi_j(x)\rho(x)dx = \delta_{ij}\] is satisfied only if the two polynomials are 
    orthonormal. For all other non-zero values, it can be deduced that the polynomials are orthogonal.
\end{itemize}

\subsection{Orthonormal Polynomial Generation}
After implementing the code for orthonormal polynomial generation up to 8 degrees, we got the following matrix for the
expected values $\mathbb{E}[\phi_i(x)\cdot\phi_j(x)]$ for each pair of $\phi_i(x)$ and $\phi_j(x)$ using uniform 
distribution:
\begin{table}[H]
\centering
\makebox[\textwidth]{
    \begin{tabular}{|c|c|c|c|c|c|c|c|c|c|}
        \hline
        \diagbox{j}{i} & 1 & 2 & 3 & 4 & 5 & 6 & 7 & 8 & 9 \\
        \hline
        1 & 1.000e+00 & 0.000e+00 & 0.000e+00 & 0.000e+00 & 4.441e-16 & 0.000e+00 & -3.553e-15 & 0.000e+00 & 3.553e-15 \\ 
        \hline
        2 & 0.000e+00 & 1.000e+00 & 0.000e+00 & 0.000e+00 & 0.000e+00 & -3.553e-15 & 0.000e+00 & 0.000e+00 & 0.000e+00 \\
        \hline
        3 & 0.000e+00 & 0.000e+00 & 1.000e+00 & 0.000e+00 & -8.882e-16 & 0.000e+00 & -3.553e-15 & 0.000e+00 & 1.421e-14 \\
        \hline
        4 & 0.000e+00 & 0.000e+00 & 0.000e+00 & 1.000e+00 & 0.000e+00 & 3.553e-15 & 0.000e+00 & 1.421e-14 & 0.000e+00 \\
        \hline
        5 & 4.441e-16 & 0.000e+00 & -8.882e-16 & 0.000e+00 & 1.000e+00 & 0.000e+00 & 1.421e-14 & 0.000e+00 & -5.684e-14 \\
        \hline
        6 & 0.000e+00 & -3.553e-15 & 0.000e+00 & 3.553e-15 & 0.000e+00 & 1.000e+00 & 0.000e+00 & 8.527e-14 & 0.000e+00 \\
        \hline
        7 & -3.553e-15 & 0.000e+00 & -3.553e-15 & 0.000e+00 & 1.421e-14 & 0.000e+00 & 1.000e+00 & 0.000e+00 & 4.547e-13 \\
        \hline
        8 & 0.000e+00 & 0.000e+00 & 0.000e+00 & 1.421e-14 & 0.000e+00 & 8.527e-14 & 0.000e+00 & 1.000e+00 & 0.000e+00 \\
        \hline
        9 & 3.553e-15 & 0.000e+00 & 1.421e-14 & 0.000e+00 & -5.684e-14 & 0.000e+00 & 0.000e+00 & 0.000e+00 & 1.000e+00 \\
        \hline
    \end{tabular}
}
\caption{Expectation values for the uniform distribution}\label{tab:uniform}
\end{table}
Similarly for the normal distribution case, we obtained the following results:
\begin{table}[H]
    \centering
    \makebox[\textwidth]{
        \begin{tabular}{|c|c|c|c|c|c|c|c|c|c|}
            \hline
            \diagbox{j}{i} & 1 & 2 & 3 & 4 & 5 & 6 & 7 & 8 & 9 \\
            \hline
            1 & 1.000e+00 & 0.000e+00 & 1.421e-14 & 5.684e-14 & 1.819e-12 & -1.455e-11 & -8.731e-11 & 0.000e+00 & -8.149e-10 \\
            \hline
            2 & 0.000e+00 & 1.000e+00 & -2.274e-13 & 5.457e-12 & -1.819e-11 & -4.366e-11 & 2.095e-09 & 3.725e-09 & -4.470e-08 \\
            \hline
            3 & 1.421e-14 & -2.274e-13 & 1.000e+00 & 6.185e-11 & 7.567e-10 & 6.985e-10 & -3.260e-08 & -3.297e-07 & -8.270e-07 \\
            \hline
            4 & 5.684e-14 & 5.457e-12 & 6.185e-11 & 1.000e+00 & -5.006e-09 & -5.960e-08 & -1.863e-07 & -3.397e-06 & -2.146e-05 \\
            \hline
            5 & 1.819e-12 & -1.819e-11 & 7.567e-10 & -1.281e-09 & 1.000e+00 & -6.109e-07 & -3.055e-06 & -2.170e-05 & 9.727e-05 \\
            \hline
            6 & -1.455e-11 & -4.366e-11 & 4.424e-09 & -4.470e-08 & -1.341e-07 & 1.000e+00 & -3.672e-05 & -2.575e-05 & -1.274e-03 \\
            \hline
            7 & -8.731e-11 & 2.095e-09 & -1.770e-08 & 7.674e-07 & -3.055e-06 & -3.672e-05 & 1.000e+00 & 5.493e-04 & -8.484e-03 \\
            \hline
            8 & 0.000e+00 & 3.725e-09 & -9.127e-08 & -5.960e-08 & 3.934e-05 & -1.478e-04 & -1.282e-03 & 1.011e+00 & -6.653e-03 \\
            \hline
            9 & -8.149e-10 & -4.470e-08 & -1.118e-07 & 1.669e-05 & -1.469e-04 & 6.790e-04 & -6.714e-04 & 1.887e-01 & 2.402e-01 \\
            \hline
        \end{tabular}
    }
    \caption{Expectation values for the normal distribution}\label{tab:normal}
    \end{table}
As can be seen in both table \ref{tab:uniform} and \ref{tab:uniform}, diagonal entries (i.e. entries where the Kronecker
delta $\delta_{ij}=1)$ show an expectation value of 1 as expected, while non-diagonal entries report either 0 or values 
with very small order of magnitudes, possibly due to floating point calculations. Only exception is the $i=j=9$ case for 
the normal distribution, where the value is the largest among other elements in the row but still not equal to 1. It can
be attributed to the same reason that there is more non-zero off-diagonal entries in the normal distribution case, yet
we cannot tell the exact reason why.
\end{document}
